\documentclass[a4paper,12pt]{report}

\usepackage{graphicx}
\usepackage{amsmath}
\usepackage{fullpage}
\usepackage[parfill]{parskip}
\usepackage{amssymb}
\usepackage{alltt}
\usepackage{listings}

\newcommand{\degree}{\ensuremath{^\circ}}
\newcommand{\unit}[1]{\ensuremath{\, \mathrm{#1}}}

\begin{document}
\title{Subnetting}
\author{Kimberley Manning}
\date{21 September 2012}
\newpage

\textbf{Question 1}

11111111 11111111 10000000 00000000

$17 network -> 32-17 = 15 bits for host --> 2^15 total host IDs, but 2 are reserved for the network and broadcast addresses so 2^15 - 2 = 32766 IP addresses$

\textbf{Question 2}

131.10.255.255/16 --> all host ID bits are set, broadcast address
131.10.255.256/23 --> can't represent 256 in 8 bits
29.23.45.16/33 --> can't have 33 network bits in 32 bits
127.1.1.1/8 --> starting with 127 is a loopback address
131.1.1.1.1/24 --> more than 32 bits
131.10.255.254/17 --> okay?
131.0.0.77/32 --> no host ID part

\textbf{Question 3}

device A and D have same IP

\textbf{Question 4}

straightforward.

\textbf{Question 5}

a) netmask: 255.255.255.11100000
network ID: 192.168.0.0
host ID: 14??
min IP: 192.168.0.11100001 (ending in 0 is network address)
max IP: 192.168.0.11111110 (ending in 1 is broadcast address)
number of hosts: 2tothe5 minus 2 = 30 hosts

etc for rest.

\textbf{Question 6}

\begin{alltt}
PING google.com (74.125.24.102) 56(84) bytes of data.

--- google.com ping statistics ---
5 packets transmitted, 0 received, 100% packet loss, time 3999ms
\end{alltt}


\end{document}

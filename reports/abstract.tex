\chapter{Abstract}

Software-defined networking (SDN) is a relatively new concept in networking which aims to give network operators greater programmability and control over their networks, through separation of the control and data planes. A centralised controller sends network control packets to other switches on the network, usually along a separate, parallel control network. With this new, centralised view of the network, it is possible to apply well-understood mathematical techniques from optimisation and operations research, possibly improving routing efficiency.

This thesis presents a framework to facilitate comparison of different routing metrics. The framework includes a controller, consisting of modules for topology discovery, flow statistics and routing control, and an experiment module to run network experiments on virtual networks, including several prewritten topologies. Three routing metrics (shortest path, widest path and residual capacity) are implemented.

Additionally, this thesis presents the results of network experiments performed using this framework, comparing the performance of the three metrics above on two topologies with different flow patterns. The residual capacity metric, which attempts to find a globally-optimal solution, improved on the performance of shortest path by up to X\% in some trials, but performed significantly worse when no solution could be found. Widest path improved on shortest path by up to Y\% consistently.

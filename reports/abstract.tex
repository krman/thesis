\chapter{Abstract}

Software-defined networking (SDN) is a relatively new concept in networking which aims to give network operators greater programmability and control over their networks, through separation of the control and data planes. The way this is most commonly implemented involves a centralised controller which sends network control packets to other switches on the network, usually on a separate, parallel control network. 

With this new, centralised view of the network, it is possible to apply well-understood mathematical techniques from optimisation and operations research. solve a type of problem known as the multicommodity flow problem. 

Mesh networks, however, are inherently distributed: this contradicts several key assumptions made in most discussions of SDN about the network architecture, especially for wireless meshes. A centralised, single-point-of-failure controller seems not to meet the requirements of the system, but the other benefits of software-defined control are such that several different groups have attempted to design hybrid controllers which do, to varying degrees and with sometimes limited scope. This thesis will compare these approaches and, using insights from this process, present a new controller.

The goal of this thesis is to implement a central controller to do optimal routing on a wireless mesh network, given the network topology and link information. The controller will be benchmarked against a selection of existing approaches to routing in wireless mesh networks. Reproducible network tests will be performed using the network simulator Mininet; the controller will be implemented using the controller framework POX.

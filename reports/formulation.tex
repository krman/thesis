\documentclass[12pt,a4paper]{article}
\usepackage{textgreek}

\usepackage{amsfonts}
\usepackage{amsmath}
\usepackage{setspace}
\usepackage{fullpage}
\usepackage{parskip}
\onehalfspacing
\setlength{\parindent}{0cm}


%\chapter{Algorithm Design}

%end to end flow is important

%aggregation of flows is important

%some smart dudes \cite{bonsma:ufp}, \cite{anag:mazing} and \cite{chakrabarti:ufp}

%\section{Considerations}
%tcp imposes two significant restrictions:

%\begin{enumerate}
	%\item everything goes to shit if you send things down multiple paths
	%\item tcp slows down its sending rate if you lose packets, so demand changes
%\end{enumerate}

\begin{document}
\section{Formulation}
The following formulation of the problem attempts to maximise the minimum spare capacity over all links, and is adapted from a formulation by Walkowiak \cite{walkowiak:residual}.

%\newpage
Begin with a network $G$ with vertices $V$ and edges $E$, with edge capacities $c : E \rightarrow \mathbb{R}^+$. The set $P$ comprises $p$ commodities to be routed through $G$, where the $i$th commodity corresponds to a flow with a source $s_i \in V$, destination $t_i \in V$ and demand $d_i \in \mathbb{R}^+$. For each $i \in P$, there are $l_i$ possible routes between $s_i$ and $t_i$, so define the set $\Pi_i = \{\pi_i^k : k = 1, ..., l_i\}$ to represent these.

In the unsplittable flow problem, each commodity must be routed entirely along one path. Each potential path $\pi_i^k$ is therefore associated with a corresponding variable $x_i^k \in \{0,1\}$, indicating whether that path was chosen for commodity $i$. The total traffic across edge $j \in E$ must be less than the capacity of the link. The constant $a_{ij}^k \in \{0,1\}$ indicates whether path $\pi_i^k$ uses edge $j \in E$.

The problem can be expressed as a linear program as follows.

\begin{equation*}
	\text{max } z \text{ s.t.} \\
\end{equation*}
\begin{align}
	\sum_{\pi_i^k \in \Pi_i} x_i^k = 1 \hspace{1cm} &\forall i \in P \\
	x_i^k \in {0,1} \hspace{1cm} &\forall i \in P; \pi_i^k \in \Pi_i \\
	f_j = \sum_{i \in P} \sum_{\pi_i^k \in \Pi_i} a_{ij}^k x_i^k d_i \hspace{1cm} &\forall j \in E \\
	f_j \leq c_j \hspace{1cm} &\forall j \in E \\
	z \leq c_j - f_j \hspace{1cm} &\forall j \in E
\end{align}

In short: the goal is to maximise $z$. Constraint (3.5) defines $z$ so that for every edge (or link), the spare capacity on that link is at least $z$. The total traffic routed through each link, defined in (3.3), must be less than its capacity, by (3.4). Each possible route for a given commodity is either selected or not, and only one route per commodity is selected, by (3.2) and (3.1) respectively.

%\newpage
%when i get home, write out how to formulate this in the standard form expected by algorithms and pulp and stuff. eg there was that stackoverflow question saying how to change (3.3) from the weird 3 variable thing into a standard linear equation constraint, coz 2 of the variables are binary, so you can fix it. yeah. 2 variables are binary and 1 is constant defined at the start (well, as constant as demand can be).

\end{document}
book

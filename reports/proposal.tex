\documentclass[pdftex,12pt,a4paper]{article}

\usepackage{fullpage}
\usepackage{setspace}
\usepackage[usenames,dvipsnames]{xcolor}
\usepackage{framed}
\usepackage[pdftex]{graphicx}
\usepackage{booktabs}
\usepackage{float}
\usepackage{tikz}
\usepackage{tabularx}

\doublespacing
\setlength{\parindent}{1.5cm}

\begin{document}

\thispagestyle{empty}
\begin{center}
	\vspace*{35mm}
	{\huge\bf
		Optimal Mesh Routing \\
		in Software Defined Networks \\
	}

	\vspace{20mm}
	{\Large 
		METR4900 Thesis Proposal \\
	}

	\vspace{20mm}
	Kimberley Manning / 41966350 \\
	Supervisor: Marius Portmann
\end{center}

\newpage
\thispagestyle{empty}
\tableofcontents
\listoftables

\newpage
\pagenumbering{arabic}
\section{Overview}
The goal of this thesis is to implement a central controller to do optimal routing on a wireless mesh network, given the network topology and link information. The controller will be benchmarked against a selection of existing approaches to routing in wireless mesh networks. Reproducible network tests will be performed using the network simulator Mininet; the controller will be implemented using the controller framework POX.

\subsection{Goals}
The project's goals can be summarised as follows:

\begin{enumerate}
	\item Develop a set of topologies and events to use in testing
	\item Benchmark the performance of major approaches to mesh routing
	\item Design, implement and test a new controller in POX
\end{enumerate}

\subsection{Relevance}
Software-defined networking (SDN) is a relatively new concept in networking which aims to give network operators greater programmability and control over their networks, through separation of the control and data planes. The way this is most commonly implemented involves a centralised controller which sends network control packets to other switches on the network, usually on a separate, parallel control network. 

Mesh networks, however, are inherently distributed: this contradicts several key assumptions made in most discussions of SDN about the network architecture, especially for wireless meshes. A centralised, single-point-of-failure controller seems not to meet the requirements of the system, but the other benefits of software-defined control are such that several different groups have attempted to design hybrid controllers which do, to varying degrees and with sometimes limited scope. This thesis will compare these approaches and, using insights from this process, present a new controller.

\newpage
\section{Background}
\subsection{Software-Defined Networking}
Software-defined networking is a set of philosophies and concepts on how networking should be done. Crucially, networking should be programmable. The core idea is separation of the control and data planes; specifically, separation in a way that is standards-compliant and vendor-independent \cite{onf:sdn}. Although the movement is not intrinsically tied to any specific technologies, the OpenFlow protocol \cite{onf:switch140}, which is used to communicate between the control and data planes, is quite strongly linked to it.

McKeown describes the networking industry as vertically integrated and proprietary, with little innovation \cite{mckeown:sdn}, comparing this to the state of computing prior to the adoption of high-level operating systems using standard instruction sets to communicate with the hardware. Similarly, the control plane in SDN can be seen as a network operating system, with OpenFlow as the instruction set for communicating with switches. This echoes the development towards greater abstraction evidenced by the move towards flow- over packet-based networking and the rise of quality-of-service considerations. Such requirements can be difficult to meet without some control over switch forwarding tables.

OpenFlow was originally conceived \cite{mckeown:openflow} as a research tool to enable academics to test new protocols easily and receive rapid feedback, while allowing vendors to continue to protect the inner workings of their switches. More recently \cite{onf:sdn} the focus has shifted to large datacenters and commercial networks with complex routing requirements. In \cite{mckeown:sdn}, McKeown refers to the  ``ossified network'': due to the black-box nature of modern network components such as switches and routers, experimentation is not encouraged and researchers and network administrators must stick to standard protocols until new ones are supported by vendors. This process can take a long time and the feedback cycle is slow. In contrast, SDN offers the ability to easily implement new protocols on real networks, and programmatically and automatically test and monitor many more aspects of the system.

\newpage
\subsection{Mesh Networks}
A wireless mesh network (WMN) is a distributed network made up of collection of routers or switches connected wirelessly.
This contradicts certain assumptions about the way SDN is implemented. Mesh networks are inherently decentralised with no single point of failure, but most SDN implementations have a central controller. There is no separate control network, so all network control messages must pass through the same wireless ports as regular traffic, and communication resources are often more limited than in regular networks \cite{detti:wmsdn}. Any part of the mesh may drop out and hence lose contact with the controller at any point, so fallback needs to be implemented. Mendonca et al \cite{mendonca:hetero} outline some specific factors to consider when designing for heterogenous environments, many of which apply to WMNs as well: buffer size, mobility, impermanence and network interoperability.

However, the difficulty of managing WMNs due to their distributed nature means that SDN is seen as a desirable solution. All of SDN's usual advantages, such as programmability, automation and so on, still apply. In general, SDN allows more of the meaningful information available at the low level to be available at a higher level for the purposes of network management \cite{mendonca:hetero, dely:wmn}, and content can be more efficiently stored and delivered. For example, approaches such as that in \cite{handigol:asterix}, where load-balancing is performed in a distributed fashion as a network primitive, can be taken. Many of these innovative approaches still require some level of centralised control, which the SDN controller provides.

\newpage
\subsection{Routing Algorithms}
\subsubsection{Traditional Mesh Routing}
Mesh routing is inherently distributed and a number of protocols explicitly designed for distributed routing exist to address this, such as AODV, B.A.T.M.A.N. and OLSR. These protocols do not rely on centralised control, and make design decisions based on the high turnover and fragmentation in such networks \cite{dely:wmn, detti:wmsdn}. Of these, Optimized Link State Routing (OLSR) \cite{rfc3626} is frequently used as a backup protocol for SDN-based solutions so it is worth examining. Generally, link-state routing protocol require that nodes each maintain a copy of the network topology database. In OLSR, compared to the similar IP routing protocol OSPF \cite{rfc5340}, only a subset of nodes store the data, and the protocol does not attempt to ensure that all nodes are always up-to-date, but simply floods the network often enough that each node's database is updated reasonably often.

\subsubsection{Software-Defined Mesh Routing}
Software-defined networks rely on two concepts: flow-based routing and the existence of a central controller. Flow-based routing means routing based on flows (linked series of related packets such as all TCP traffic, all traffic from one MAC address or to a particular IP subnet). This concept was reasonably well-developed \cite{wellons:oblivious,wang:routing} before SDN.

The other element of SDN is the existence of a central network OS which can communicate with all nodes. This raises the obvious objection that in a mesh network centralisation is not ideal; however, this is not centralised routing where all packets must pass through the controller. \cite{handigol:asterix} makes the distinction between ``logically centralised'' and ``distributed through the network''. Once rules are installed packet forwarding happens at individual switches as usual. The controller can push an initial set of rules as soon as a switch connects to it if desired. Later, when the first packet of a new kind of flow arrives at a switch, it encapsulates the packet and sends it to the controller. The controller then installs the appropriate rules on the appropriate switches \cite{mckeown:sdn}. Depending on configuration (fail standalone mode or fail secure mode) these rules can remain in place if the controller connection is lost, or the switch can drop all packets \cite{onf:switch140}.

Dely et al \cite{dely:wmn} implemented an SDN controller and tested it on KAUMesh using the metrics of forwarding performance, amount of control traffic and rule activation time (time between the first packet of a new flow arriving, and that packet being forwarded onwards). They found that while SDN drastically increased ease of development (the problem of node mobility was solved in a few lines of code), the SDN controller performed slightly worse on their network than existing algorithms such as OLSR. The performance was within acceptable limits for their small-scale test, but they noted that scalability could be an issue when deploying on larger networks.

Detti et al \cite{detti:wmsdn} found improved user performance using their software toolkit (wmSDN) in a traffic engineering application compared to a traditional network. The authors particularly focus on the fine-grained control SDN allows, which mean more advanced traffic routing algorithms could be implemented on the network. As a solution to the problem of centralised control in an often-fragmented network, they used OLSR to route both control traffic and data traffic when no connection to the controller was available.

One approach mentioned in \cite{wellons:augmenting} and \cite{dai:dynamic} is to model flow-based networking as a multi-commodity network flow problem \cite[pp. 862--863]{cormen:algorithms}. Flows are modelled as commodities moving between various source and sink nodes in the network. This is an optimisation problem where the objective function can be adjusted depending on the desired outcomes, such as maximising overall throughput or maximising worst-case performance. The authors of \cite{wellons:augmenting} note that such approaches can be sensitive to accurate predictions of demand, but had success merging it with more dynamic heuristic algorithms. The multi-commodity network flow model is the initial direction this thesis will take.

\newpage
\subsection{Protocols and Tools}
\subsubsection{OpenFlow}
OpenFlow \cite{onf:switch140} is a protocol for communication between switches and a controller. It is based on the existing concept of match-action pairs at different levels of granularity \cite{mckeown:sdn}. The developers identified the set of actions that that most switch vendors had in common. The set is extensible, but there is a minimum requirement that all switches must meet: forward packets to a port, encapsulate packets and send to controller, or drop packets. All switches must also provide some method of separating production traffic from experimental, either by running separate VLANs or by implementing an additional action.

\subsubsection{Mininet}
Mininet is a network emulator which uses container-based emulation \cite{handigol:mininet} to create a virtual network of hosts and switches. Mininet makes it simple to run reproducible network experiments under realistic conditions, with topologies and behaviour programmable in Python. The resources allocated to each host can be controlled and monitored, and benchmarking \cite{handigol:benchmarks} has generally shown that the performance and timing characteristics are accurate. OpenFlow-enabled switches such as OpenVSwitch can be used in the simulations so the system is ideal for SDN research.

\subsubsection{POX}
There are a number of controller frameworks today which allow programmers to send OpenFlow messages to switches in the network; selection in the first case is based on familiarity with the development language used. One such framework, based on Python, is POX \cite{onl:pox}, developed at Stanford primarily for ease of research over speed and performance. This is acceptable if performance of the network is not bound by the controller, at least for initial development; this can be verified if necessary.

\newpage
\section{Project Plan}
\subsection{Contribution}
The goal of this thesis is to implement a controller to do optimal routing in a mesh network, with comparable or better performance than existing solutions available, assuming that the controller has topology and link capacity information and a route to each switch in the network. The problem can be initially modelled as a multi-commodity flow problem and solved using standard optimisation techniques.

In order to assess the performance of the controller, a set of meaningful network tests will be devised and the controller's performance benchmarked against that of a selection of existing techniques for software-defined networking in wireless mesh networks.

\subsection{Methodology}
The project can be divided into a number of steps:

\begin{enumerate}
\item Compile an overview of common approaches, and select a representative sample which can be experimentally reproduced; implement a working version of each selected approach
\item Mathematically model the proposed new controller; implement it in POX
\item Explicitly define the performance objectives and benchmarking metrics; design a set of topologies to capture performance differences between approaches; outline a range of network events to simulate in Mininet 
\item Conduct benchmarking of existing approaches and new controller; note relative performance on various metrics, and improve controller based on benchmarking results
\end{enumerate}

A suggested schedule including dates of assessment is outlined in Table \ref{table:schedule}.

\subsection{Schedule}

\begin{table}[H]
	\begin{center}
		\begin{tabular}{@{}rllr@{}}
			\toprule
			& Task & Duration & Completion\\
			\midrule
			& Establish project definition and scope & 1 week & 6 March \\
			& Set up environment (Mininet,  POX) & 1 week & 14 March \\
			\emph{Assessment} & Project proposal & 2 weeks & 27 March \\
			& Set up test suite (topologies, network events) & 1 week & 4 April \\
			& Implement selection of existing designs & 1 week & 11 April \\
			& Model proposed design & 1 week & 18 April \\
			& Begin implementing proposed design & 2 weeks & 2 May \\
			& Benchmark new controller against existing & 1 week & 9 May \\
			\emph{Assessment} & Progress seminar & 2 weeks & 19-23 May \\
			\addlinespace
			& \emph{Coursework examinations/semester break} \\
			\addlinespace
			& Write up current progress in thesis report & 2 weeks & 8 Aug\\
			& Revise test suite as necessary & 2 weeks & 22 Aug\\
			& Iteratively develop/test new controller & 5 weeks & 26 Sept\\
			& Gather final benchmarking results & 2 weeks & 10 Oct \\
			\emph{Assessment} & Project demonstration & 2 weeks & 20-24 Oct \\
			& Respond to demonstration feedback & 1 week & 31 Oct \\
			\emph{Assessment} & Thesis report & 3 weeks & 10 Nov \\
			\bottomrule
		\end{tabular}
		\caption{Proposed schedule outlining estimated duration and completion dates}
		\label{table:schedule}
	\end{center}
\end{table}

\subsection{Risk Assessment}
\subsubsection{Occupational Health and Safety}
This project requires work which can be completed in low-risk computing laboratories, which are covered by general OHS laboratory rules.

\subsubsection{Project Risks and Response}
Table \ref{table:risks} identifies the potential risks to the project, listed in order of severity from 1 (least critical) to 5 (most critical). Mitigation strategies for each risk are listed in Table \ref{table:mitigation}.

\hspace{0.5cm}
\begin{table}[H]
	\begin{center}
		\begin{tabular}{@{}rlllr@{}}
			\toprule
			\# & Risk & Likelihood & Impact \\
			\midrule
			1 & Loss of working environment & low & low \\
			2 & Loss of research data & low & medium \\
			3 & Illness during thesis & medium & medium \\
			4 & Major scope changes & high & medium \\
			5 & Falling behind schedule & high & high \\
			\bottomrule
		\end{tabular}
		\caption{Likelihood and severity of potential risks to the project}
		\label{table:risks}
	\end{center}
\end{table}

\begin{table}[H]
	\begin{center}
		\begin{tabularx}{\textwidth}{p{.02\textwidth}X}
			\toprule
			\# & Risk: Action \\
			\midrule
			1 & Loss of working environment:
			maintain list of software versions in use along with install notes so environment can be quickly recreated. \\
			\addlinespace
			2 & Loss of research data:
			commit all raw data (reports, slides, code, test results) to remote git repository; back up working directory to Dropbox/EAIT fileserver. \\
			\addlinespace
			3 & Illness during thesis:
			inform supervisor as soon as practicable to discuss impact on relevant assessment; if necessary obtain medical certificate and request extension. \\
			\addlinespace
			4 & Major scope changes:
			discuss suitable direction with supervisor; determine what prior work can be integrated into new report; reassess project schedule. \\
			\addlinespace
			5 & Falling behind schedule:
			conduct weekly progress meetings with supervisor with deliverables; regularly check progress against original schedule. \\
			\addlinespace
			\bottomrule
		\end{tabularx}
		\caption{Risk mitigation strategies for potential project risks}
		\label{table:mitigation}
	\end{center}
\end{table}

\newpage 
\pagenumbering{gobble}
\bibliographystyle{abbrv}
\bibliography{bib}

\end{document}

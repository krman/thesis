\documentclass[pdftex,12pt,a4paper]{article}

\usepackage{fullpage}
\usepackage{setspace}
\usepackage[usenames,dvipsnames]{xcolor}
\usepackage{framed}
\usepackage[pdftex]{graphicx}
\usepackage{booktabs}
\usepackage{float}
\usepackage{tikz}

\doublespacing
\setlength{\parindent}{1.5cm}

\begin{document}

\thispagestyle{empty}
\begin{center}
	\vspace*{35mm}
	{\huge\bf
		Optimal Mesh Routing \\
		in Software Defined Networks \\
	}

	\vspace{20mm}
	{\Large 
		METR4900 Thesis Proposal \\
	}

	\vspace{20mm}
	Kimberley Manning / 41966350 \\
	Supervisor: Marius Portmann
\end{center}

\newpage
\thispagestyle{empty}
\tableofcontents
\listoftables

\newpage
\pagenumbering{arabic}
\section{Overview}
goal of this project is to benchmark the performance of a selection of control approaches for sdn in wireless mesh networks, and using insights gleaned from this process, implement a controller to do optimal routing on a mesh network. reproducible network tests will be performed using the network simulator Mininet. the controller will be implemented using the controller framework POX.

\subsection{Goals}

the central aim of this project is to implement a controller to do optimal routing on a mesh network. in aid of this the project has three subgoals.

\begin{enumerate}
	\item develop a set of topologies to use in testing
	\item benchmark the performance of major approaches to control
	\item implement a controller in POX
\end{enumerate}

\subsection{Relevance}

sdn is a relatively new concept in networking which aims to give network operators greater programmability and control over their networks, through separation of the control and data planes. the way this is most commonly implemented involves a centralised controller which sends network control packets to other switches on the network, usually on a separate, parallel control network. 

a number of attempts have been made to use sdn in wireless mesh networks. mesh networks are inherently distributed: this contradicts several key assumptions about the network architecture, especially for wireless meshes. trying to integrate a centralised, single-point-of-failure controller into a mesh goes against the requirements of the system. different groups have tried to deal with this problem in different ways.

\newpage
\section{Background}
\subsection{Software-Defined Networking}
\emph{Provide an overview of some of the key areas of research in SDN. What are researchers currently working on, and what are considered the key research challenges?}

software-defined networking (sdn) is a set of philosophies/concepts on how networking should be done. networking should be programmable. core idea is separation of the control and data planes, and more specifically, separation in a way that's standards-compliant and vendor-independent. \cite{onf:sdn} not linked to any specific technologies although some like openflow are quite strongly (not intrinsically) linked to it. 

sdn is often mentioned in conjunction with the openflow protocol ... originally conceived as a research tool to enable academics to test new protocols easily and receive rapid feedback. \cite{mckeown:openflow} mckeown refers to the "ossified network": due to the black-box nature of modern network components such as switches and routers, experimentation is not encouraged and researchers must stick to standard protocols until new ones are supported by vendors. this process can take a long time and the feedback cycle is slow. the networking industry is vertically integrated and proprietary, with little innovation. \cite{mckeown:sdn}

idea of the controller as a kind of network OS. openflow is the instruction set.

note that the original paper \cite{mckeown:openflow} was massively into on-campus research applications, and the onf is really into large datacentres and stuff \cite{onf:sdn} which are both fixed, usually ethernet-based. i think the onf whitepaper talked about other examples, but they were mostly things like virtualisation where the assumption of a dedicated control plane is still there.

\subsection{Mesh Networks}
wireless mesh networks are a thing. they are distributed, often over wide geographical area, and are only connected in wireless links. this contradicts certain assumptions about the way sdn is implemented. mesh networks are inherently decentralised with no single point of failure, but most sdn implementations have a central controller. there is no separate control network, so all network control messages must be in-band. any part of the mesh may drop out and hence lose contact with the controller at any point, so fallback needs to be implemented.

on the surface they don't seem very well suited to sdn at all, what with the whole centralised controller thing. need to come up with reasons you would still want to do sdn for wmns. all the other advantages like programmability, automation etc must still apply. note that openflow lets switches be controlled by multiple controllers if necessary, or you can load-balance/fault-tolerance-ise the controllers with something else, or whatever. or you can be like \cite{handigol:asterix} and give each switch the ability to do useful things itself (they did load-balancing "as a network primitive", which is fairly cool).

\subsection{Routing Algorithms}
\subsubsection{Traditional Mesh Routing}
discussion of what makes a good mesh routing protocol. mesh routing is inherently distributed so don't force packets through centralised bits and stuff like that.

most common non-sdn mesh routing protocols. traditional routing involves routing based on individual packets (common knowledge?).

major ones marius has mentioned: olsr \cite{rfc3626}, ospf \cite{rfc5340}, obsolete \cite{rfc4813}

\begin{description}
\item[cats] love you a lot
\item[dogs] love the stars more/Prop
\item[armadillos] are very long-winded
\end{description}

\subsubsection{Software-Defined Mesh Routing}
attempts to do mesh routing in sdn context. flow-based routing means routing based on flows (linked series of related packets eg all tcp traffic, all traffic from one mac address or to a particular ip subnet).

commonly seen as a solution to 

note that this is not "centralised" routing. once rules are installed packet forwarding happens at individual switches exactly as it always has, at line rate (right?). the controller can push an initial set of rules as soon as a switch connects to it if desired. depending on configuration (fail standalone mode or fail secure mode) these rules can remain in place if the controller connection is lost, or the switch can drop all packets. \cite{handigol:asterix} makes the distinction between "logically centralised" and "distributed through the network". so, fine to have centralised controller if you can talk to that. 

header space maths \cite{kazemian:header} originally in \cite{lakshman:geo}

flow-based networking can be modelled as a multi-commodity flow problem

this is literally stolen, rephrase this.

Given a flow network $G(V,E)$, where edge $(u,v)$ has capacity $c(u,v)$, flow can be assigned as follows:

\vspace{-0.5cm}
\begin{equation}
\sum\limits_{w \in V} f_i(s_i,w) = \sum\limits_{w \in V} f_i(w,t_i) = d_i
\end{equation}

\subsection{Protocols and Tools}
\subsubsection{OpenFlow}
spec? \cite{onf:switch140}. 

i believe (find a source, this was from a blog post or something) that the new version of the spec \cite{onf:switch140} has more support for experimental stuff and people had some comments about whether this was good or bad for open standards

openflow is for communicating between switches and controller. 

allows switch to be controlled by multiple controllers \cite{mckeown:openflow}

\subsubsection{Mininet}
talk about:

what mininet is and how it works with the containers and stuff \cite{handigol:mininet}

mininet benchmarks \cite{handigol:benchmarks}

can use pretend openflow-enabled switches in the simulations, like openvswitch (no idea how to cite this)

\subsubsection{POX}
\emph{Give an overview and brief comparison of the key SDN controllers that are currently available, commercial and non-commercial.}

maybe list the major controllers, at least the related ones? pox is in python which is why i like it

designed more for ease of research than speed/performance (NOX (C++) is that) (http://www.noxrepo.org/2012/03/introducing-pox/, dunno about source for this). this is okay if performance at the controller isn't the bottleneck. may need to actually confirm this is the case.

\newpage
\section{Project Plan}
\subsection{Contribution}
problem: implement a controller to do optimal routing in a mesh network

assumptions: controller has topology and link capacity info, and a route to each switch in the network

constraints: want packets in a flow to arrive in order, so all packets within a flow. define flow in this context to mean between a single source/dest

solution: model as multi-commodity flow problem.

\subsection{Methodology}

as stated earlier the project involves comparing the performance of existing approaches to sdn for wmns, and delivering a controller in pox to do optimal routing, and compare this to the benchmarked performance.

things to do in order to achieve this:

overview of common approaches (research)

note common "features" of these approaches and what sort of topologies would best capture them. well, what sort of topologies would best allow us to differentiate between different approaches.

in addition to topologies: come up with range of network events to simulate as well. mininet allows simulation of stuff like link up/down, has good wireless stuff eg link slowly moving out of range.

do maths (at some point)

implement a working version of each selected approach

implement controller, note performance on various metrics, improve based on results

all of these steps including dates of assessment are summarised in Table \ref{table:schedule}

\begin{table}[H]
	\begin{center}
		\begin{tabular}{@{}rllr@{}}
			\toprule
			& Task & Duration & Completion\\
			\midrule
			& Establish project definition and scope & 1 week & 6 March \\
			& Set up environment (Mininet,  POX) & 1 week & 14 March \\
			\emph{Assessment} & Project proposal & 2 weeks & 27 March \\
			& Project proposal & weeks & 10 March \\
			& Project proposal & weeks & 10 March \\
			& Project proposal & weeks & 10 March \\
			\emph{Assessment} & Progress seminar & 2 weeks & 19-23 May \\
			& Project proposal & weeks & March \\
			& Project proposal & weeks & March \\
			& Project proposal & weeks & March \\
			& Project proposal & weeks & March \\
			& Project proposal & weeks & March \\
			\emph{Assessment} & Project demonstration & 2 weeks & 20-24 Oct \\
			& Respond to demonstration feedback & 1 week & 31 Oct \\
			\emph{Assessment} & Thesis report & 3 weeks & 10 Nov \\
			\bottomrule
		\end{tabular}
		\caption{Proposed schedule outlining estimated duration and completion dates}
		\label{table:schedule}
	\end{center}
\end{table}

\subsection{Risk Assessment}
\subsubsection{Occupational Health and Safety}
This project requires work which can be completed in low-risk computing laboratories, which are covered by general OHS laboratory rules.

\subsubsection{Project Risks}

\begin{table}[H]
	\begin{center}
		\begin{tabular}{@{}rlllr@{}}
			\toprule
			\# & Risk & Likelihood & Impact & Ranking \\
			\midrule
			1 & lose all my datas & low & low & 5 \\
			2 & lose primary working environment & low & medium & 4 \\
			3 & illness during thesis & medium & low & 3 \\
			4 & fall behind schedule & high & medium & 1 \\
			5 & scope changes & high & medium & 2 \\
			\bottomrule
		\end{tabular}
		\caption{Likelihood and severity of potential risks to the project}
	\end{center}
\end{table}

\subsubsection{Risk Response}

\begin{table}[H]
	\begin{center}
		\begin{tabular}{@{}rp{6cm}@{\hspace{1cm}}p{8cm}@{}}
			\toprule
			\# & Risk & Action \\
			\midrule
			1 & lose all my datas & 
			backup strategy: git repository hosted "online". working directory backed up to dropbox. plain text files. \\
			\addlinespace
			2 & lose primary working environment &
			maintain list of software versions in use and install notes so environment can be quickly recreated \\
			\addlinespace
			3 & illness during thesis &
			inform supervisor asap to discuss impact on relevant assessment. if necessary obtain medical cert and request extension. \\
			\addlinespace
			4 & fall behind schedule/failure to maintain schedule &
			weekly progress meetings with supervisor with deliverables \\
			\addlinespace
			5 & scope changes &
			discuss suitable direction with supervisor \\
			\addlinespace
			\bottomrule
		\end{tabular}
		\caption{Risk mitigation strategies for potential project risks}
	\end{center}
\end{table}

\newpage \bibliographystyle{abbrv}
\bibliography{bib}

\end{document}

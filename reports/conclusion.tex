\chapter{Conclusion}
\label{ch:conclusion}
The project has successfully achieved both of its aims: to create a framework to allow evaluation of routing metrics for multicommodity routing, and to perform basic analysis using the framework and report the results.

The primary software contribution of this thesis is a configurable SDN controller and experiment framework for analysis and testing. This software is supported by a set of objective functions and Mininet topologies (as discussed in this report), as well as sample experiment scripts used to produce the graphs, and various helper functions. Documentation in the form of a user guide and API reference have also been produced and are available on the companion disk.

Analysis of the performance of three different routing metrics was performed. This analysis was intended to be primarily to test ease of use of the software, and indeed these efforts identified improvements to overall workflow of the system. The results of the analysis appeared to support the hypothesis that global consideration of the network can increase routing efficiency, with the widest path and residual capacity metrics improving performance over shortest path by up to X\% and Y\% respectively. However, performance for the latter metric, which used linear programming to find an optimal solution, decreased significantly when no solution could be found.

\section{Future Work}
The analysis performed in this thesis was limited in scope, primarily due to it not being the main focus of this research. The number of routing algorithms studied was limited to three and the number of topologies to just two, one small and one large. Future work could use the framework as it stands to conduct more experiments using a wider range of metrics and topologies.

The flow patterns were also limited: only TCP/IP flows were used in experiments, and all flows arrived at the same time and continued to the end of the experiment. The experiment framework currently does allow flows to be started at separate, arbitrary times; however, the controller only runs route recalculation once, at a predetermined time, which effectively limits when flows can be run. Modifying the controller to run route calculations periodically would remove this restriction and allow testing of a greater and more realistic range of flow patterns.

Finally, as mentioned previously, \thesis{} is intended to be extended by other members of Marius Portmann's SDN group, intended eventually to be used for research into routing in wireless mesh networks. Work on this thesis has identified a number of issues which, while out of scope here, nevertheless deserve attention. Most notably, flow sizes are very difficult to estimate for TCP flows due to congestion control; this is a significant problem as global solutions are often sensitive to the estimated demands and bad estimates can result in inefficient placement of flows. Additionally, the current implementation cannot dynamically calculate link capacities and relies on hardcoded values; this greatly limits the application of this work to wireless networks, where link capacities can vary dynamically depending on physical location and other factors.

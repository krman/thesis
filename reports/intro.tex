\chapter{Introduction}
\label{ch:intro}

Software-defined networking (SDN) is a set of philosophies and concepts on how networks should be designed. The core idea is separation of the control and data planes; specifically, separation in a way that is standards-compliant and vendor-independent. Although the movement is not intrinsically tied to any specific technologies, the OpenFlow protocol is quite strongly linked to it. In a software-defined network, a central controller makes global decisions on routing and network policy (the `control' plane) and sends OpenFlow control messages to instruct switches to update their individual forwarding tables where necessary (the `data' plane).

OpenFlow is fundamentally a low-level network protocol. There now exist a number of projects which are designed to act as extensible SDN controllers, so that a programmer can simply use provided software APIs to construct and interpret OpenFlow messages instead of directly sending network packets. However, this still requires understanding of OpenFlow and network programming and functions as a convenience for programmers rather than a simple interface for end users.  Most descriptions of SDN architectures include a layer above the control layer, often called the `application' layer, for this reason. These systems are generally seen as a natural progression of the layer model as applied to network design.

This thesis presents \thesis{}, a framework designed to ease research into optimisation techniques as applied to centrally-controlled, software-defined networks. The goal of the framework is to allow a researcher to write a high-level implementation of a particular routing metric, given information about the network topology and flows and their corresponding demands. The controller then uses this metric to allocate routes to flows in the network, and the performance of the metric (in terms of overall resulting throughput in the network) can then be evaluated, without directly interacting with the OpenFlow protocol.

In this way, optimisation techniques can be easily applied to select routes, in the hope that such techniques will result in greater network efficiency. Three routing metrics are evaluated, representing a transition from naive to globally-optimum routing: shortest path, where the path with the fewest hops is always selected; widest path, which selects the path with the highest remaining capacity after considering the placement of existing flows; and residual capacity, where the combination of paths which maximises the minimum residual capacity over all links is selected, using linear programming. Intelligent routing metrics such as the latter require the global view of the network that SDN provides, and as such are difficult to implement on traditional networks. This thesis investigates whether such metrics result in a significant improvement over traditional, shortest-path-based routing methods.

\thesis{} is intended as the beginning of a larger project developed by Marius Portmann's SDN group for network research. This thesis focuses on the routing aspects of the system; support modules for statistics gathering and topology discovery were implemented to the extent required to make the system functional, but better implementations of these are the focus of other members of the SDN group. Eventually, the system will be used for research into routing in software-defined wireless mesh networks, which will produce extra challenges in gathering required information, but the modular structure of the controller means that replacing the support modules with updated versions should be straightforward.

\section{Aims}

There are two aims of this thesis:

\begin{enumerate}
  \item To create a framework to allow easier research, testing and comparison of routing metrics for multicommodity routing.
  \item To perform some basic analysis using the framework, both to demonstrate its use and to test the ease of use of the API.
\end{enumerate}

\section{Scope}
Three routing metrics will be evaluated. Evaluation will be performed on virtual networks using the network emulator Mininet only, with overall resulting throughput compared on two different topologies. Flows will only consist of TCP over IP traffic generated using the \texttt{iperf3} tool.

\section{Deliverables}
The following will be produced by the end of the thesis:

\begin{enumerate}
  \item A working controller consisting of POX modules
  \item Implementations of each routing metric in Python
  \item Mininet topology scripts used for testing
  \item A framework to launch experiment scripts
  \item A quantitative comparison of the performance of each metric
  \item Documentation on installation and use of the framework
\end{enumerate}

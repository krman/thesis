\chapter{Introduction}
\label{ch:intro}

Software-defined networking (SDN) is a set of philosophies and concepts on how networks should be designed. The core idea is separation of the control and data planes; specifically, separation in a way that is standards-compliant and vendor-independent. Although the movement is not intrinsically tied to any specific technologies, the OpenFlow protocol is quite strongly linked to it. In a software-defined network, a central controller makes global decisions on routing and network policy (the `control' plane) and sends OpenFlow control messages to instruct switches to update their individual forwarding tables where necessary (the `data' plane).

OpenFlow is fundamentally a low-level network protocol. There now exist a number of projects which are designed to act as extensible SDN controllers, so that a programmer can simply use provided software APIs to construct and interpret OpenFlow messages instead of directly sending network packets. However, this still requires understanding of OpenFlow and network programming and functions as a convenience for programmers rather than a simple interface for end users.  Most descriptions of SDN architectures include a layer above the control layer, often called the `application' layer, for this reason. These systems are generally seen as a natural progression of the layer model as applied to network design.

This thesis presents \thesis{}, a framework designed to ease research into optimisation techniques as applied to centrally-controlled, software-defined networks. 











The framework developed in this thesis is aimed somewhere to the middle, and a little to the side. There is an audience of people who work with algorithms, and there is an audience of people who know what OpenFlow is, and there is a skilled subset of people who are skilled at both. This software is not aimed at people who don't know how to code at all, but rather operations researchers who are skilled in linear programming in Python, actually probably just people in general who are have ideas about different metrics that could be used without going to all the effort of writing a controller and stuff. The goal of the software is to make it easy to test different methods of routing. 


\thesis{} is designed as the beginning of a larger project developed by Marius Portmann's SDN group for network research. This thesis focuses on the routing aspects of the system. Basic modules for statistics gathering and topology discovery were implemented to the extent required to make the system functional. The limitations of these modules are outlined in later chapters, but as others in the SDN group are working on them addressing them is considered out of scope of this project. Eventually, the system will be used for research into routing in software-defined wireless mesh networks, which will produce extra challenges in gathering prereq information etc but i've just written it assuming i have good data. or something.

\section{Aims}

There are two aims of this thesis:

To create a framework to allow easier research, practical testing and comparison of objective functions for linear programming in multicommodity routing.

To perform some basic analysis using the framework, both to demonstrate its use and to test the ease of use of the API.

\section{Scope}

Only care about TCP over IP traffic.

Only care about autonomous systems in the routing sense.

Only virtualised networks, not physical.

\section{Deliverables}

Produce the framework, a working version as POX modules in Python. Able to be installed easily.

Also implement a series of topologies in Mininet, a selection of objective functions, and some Python experiment scripts.

Use these to perform some basic analysis, demonstrating how to compare objective functions. Deliverable: graphs.

Full documentation including the many dependencies.

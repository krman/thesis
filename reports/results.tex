\chapter{Results}

This chapter presents the results of the analysis performed using the framework. Keep in mind that POX isn't designed to compete with the very best (cite POX's own graphs showing it's worse than NOX), but is designed for academia. Given that, none of this will be compared with existing industry-standard controllers such as Beacon or OpenDaylight (what is with all the light metaphors?).

Also all testing will be done entirely in software, no real switches will be used.

\section{Comparison of objective functions}
Analysis goes here. dunno what to call this, routing performance or whatever. something title-capsy.


\begin{table}[H]
  \begin{center}
    \begin{tabular}{@{}rllr@{}}
      \toprule
      & Task & Duration & Completion\\
      \midrule
      & Establish project definition and scope & 1 week & 6 March \\
      & Set up environment (Mininet,  POX) & 1 week & 14 March \\
      \emph{Assessment} & Project proposal & 2 weeks & 27 March \\
      & Set up test suite (topologies, network events) & 1 week & 4 April \\
      & Implement selection of existing designs & 1 week & 11 April \\
      & Model proposed design & 1 week & 18 April \\
      & Begin implementing proposed design & 2 weeks & 2 May \\
      & Benchmark new controller against existing & 1 week & 9 May \\
      \emph{Assessment} & Progress seminar & 2 weeks & 19-23 May \\
      \addlinespace
      & \emph{Coursework examinations/semester break} \\
      \addlinespace
      & Write up current progress in thesis report & 2 weeks & 8 Aug\\
      & Revise test suite as necessary & 2 weeks & 22 Aug\\
      & Iteratively develop/test new controller & 5 weeks & 26 Sept\\
      & Gather final benchmarking results & 2 weeks & 10 Oct \\
      \emph{Assessment} & Project demonstration & 2 weeks & 20-24 Oct \\
      & Respond to demonstration feedback & 1 week & 31 Oct \\
      \emph{Assessment} & Thesis report & 3 weeks & 10 Nov \\
      \bottomrule
    \end{tabular}
    \caption{Make tables that look as attractive as this one}
    \label{table:schedule}
  \end{center}
\end{table}

\begin{table}
\caption{Summary of incredible achievements}
\end{table}

\section{Performance and Scalability}
Hopefully this is good too. HAHAHA no it's not.

Chart goes here. (from poster). scalability sucks HARD

\chapter{Results}

This chapter presents the results of the analysis performed using the framework. Keep in mind that POX isn't designed to compete with the very best (cite POX's own graphs showing it's worse than NOX), but is designed for academia. Given that, none of this will be compared with existing industry-standard controllers such as Beacon or OpenDaylight (what is with all the light metaphors?).

However, a suitable control still needs to be used. Depending on the specific experiment (ie at what level you want to compare) a different one could be chosen. For these purposes the following are available:

\begin{itemize}
\item Plain mininet with switches running traditional routing algorithms (whatever these are)
\item POX's \texttt{forwarding.l2\_learning} controller
\item Shortest-path routing objective function (not PuLP/LP, but still uses rest of framework)
\end{itemize}

\section{Analysis}
Analysis goes here.

\subsection{Comparison with traditional routing}
Good I hope.

\subsection{Comparison with existing POX controllers}
Blah.

\subsection{Comparison of objective functions}
More stuff.

\begin{table}
\caption{Summary of incredible achievements}
\end{table}

\section{Performance and Test Coverage}
Hopefully this is good too.

Chart goes here.

100\% test coverage, guaranteed or your money back.

Oh yeah and all the test pass.
